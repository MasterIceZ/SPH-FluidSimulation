\documentclass[a4paper]{article}
\usepackage[a4paper, margin=1in]{geometry}
\usepackage{amsmath}
\usepackage{amsfonts}
\usepackage{framed}
\usepackage{xcolor}
\usepackage{hyperref}

\usepackage[natbib=true,style=authoryear,backend=bibtex,useprefix=true]{biblatex}
\addbibresource{ref.bib}

\title{SPH Fluid Simulation with OpenGL in C++}
\author{
  Borworntat Dendumrongkul\thanks{6632109921, 6632109921@student.chula.ac.th} \and 
  Dej Wongwirathorn\thanks{6432055421, 6432055421@student.chula.ac.th}
}
\date{May 2025}

\begin{document}
\maketitle

\section{Smooth Particle Hydrodynamics}

Smooth Particle Hydrodynamic (SPH) is a computational method for simulating fluid flows.
This method is based on the Lagrangian approach, where the fluid is represented by a set of particles that carry properties such as mass, velocity, and density.
The particles move with the fluid flow, and their properties are updated based on the interactions with neighboring particles.

\section{Project Overview}

The primary objective of this project is to implement a fluid simulation using the Largrangian method, specifically the SPH method.
The simulation will be visualized using OpenGL, allowing for real-time rendering of the fluid dynamics.
The project will be implemented in C++ and will utilize the OpenGL library for rendering.

\section{Implementation}

Our implementation of the SPH fluid simulation can be accessed at \href{https://github.com/MasterIceZ/SPH-FluidSimulation}{Github}.
The project is structured as follows:
\begin{itemize}
  \item \verb|src| - Contains the source code for the simulation.
  \item \verb|include| - Contains the library which is used in the simulation.
  \item \verb|shaders| - Contains the shaders used for rendering the simulation.
\end{itemize}

\subsection{Fluid Simulation Solver}

The fluid simulation solver is responsible for updating the density ($\rho$), pressure ($P$), velocity ($\mathbf{v}$), and position ($\mathbf{x}$) of the particles in each time step.
The solver uses the SPH equations to compute the interactions between particles and update their properties accordingly.

The main steps of the solver are as follows:
\begin{verbatim}
  For each particle i:
    Compute density using the kernel function
    Compute pressure using the equation of state
  
  For each particle i:
    Compute forces from neighboring particles
  
  For each particle i:
    Compute acceleration using the forces
    Update velocity using the computed acceleration
    Update position using the updated velocity
\end{verbatim}

\subsubsection{Smoothing Kernel}

The smoothing kernel is a function that defines the influence of a particle on its neighbors.
In SPH, the smoothing kernel is used to compute the density, pressure, and forces acting to the particles.

Our Implementation uses the following kernel functions:
\begin{itemize}
  \item \textbf{Wpoly6 kernel}:
    \begin{equation}
      \nonumber
      W_{poly6}(\mathbf{r}, h) = \frac{315}{64 \pi h^9} \begin{cases}
        (h^2 - r^2)^3 & \text{if } r < h \\
        0 & \text{otherwise}
      \end{cases}
    \end{equation}
  \item \textbf{Spiky kernel}:
    \begin{equation}
    \nonumber
      W_{spiky}(\mathbf{r}, h) = \frac{15}{\pi h^6} \begin{cases}
        (h - r)^3 & \text{if } 0 \le r \le h \\
        0 & \text{otherwise}
      \end{cases}
    \end{equation}
  \item \textbf{Viscosity kernel}:
    \begin{equation}
    \nonumber
      W_{visc}(\mathbf{r}, h) = \frac{15}{2 \pi h^3} \begin{cases}
        -\frac{r^3}{2h^3} + \frac{r^2}{h^2} + \frac{h}{2r} - 1 & \text{if } 0 \le r \le h \\
        0 & \text{otherwise}
      \end{cases}
    \end{equation}
\end{itemize}

Where $r$ is the distance between two particles, and $h$ is the smoothing length.

The smoothing length $h$ determines the range of influence of a particle on its neighbors.

\subsubsection{Computing Density and Pressure}

The density of a particle is computed using the $W_{poly6}$ kernel. By the definition of the density, we have:
\[
  \rho_i = \sum_{j}m_j W_{poly6}(\mathbf{r} - \mathbf{r}_j, h)
\]

where $m_j$ is the mass of particle $j$, $\mathbf{r}$ is the position of particle $i$, and $\mathbf{r}_j$ is the position of particle $j$.

In this implementation, we use the ideal gas equation of state to compute the pressure of the fluid:
\[
  p = k(\rho - \rho_0)
\]

where $k$ is the stiffness coefficient, $\rho$ is the density of the particle, and $\rho_0$ is the rest density of the fluid.

\subsubsection{Computing Forces}

The are two main forces acting on the particles in the simulation:
\begin{itemize}
  \item \textbf{Pressure force}:
    The pressure force is computed using the $W_{spiky}$ kernel. The pressure force acting on particle $i$ is given by:
    \begin{equation}
    \nonumber
      \mathbf{f}^{pressure}_i = -\sum_{j} m_j \cdot \frac{p_i + p_j}{2 \rho_j} \nabla W_{spiky}(\mathbf{r} - \mathbf{r}_j, h)
    \end{equation}
  \item \textbf{Viscosity force}:
    The viscosity force is computed using the $W_{visc}$ kernel. The viscosity force acting on particle $i$ is given by:
    \begin{equation}
    \nonumber
      \mathbf{f}^{viscosity}_i = \mu \cdot \sum_{j} m_j \cdot \frac{v_j - v_i}{\rho_j} \nabla^2 W_{visc}(\mathbf{r} - \mathbf{r}_j, h)
    \end{equation}
\end{itemize}

Where $\mu$ is the viscosity coefficient, $v_i$ is the velocity of particle $i$, and $v_j$ is the velocity of particle $j$.

Total force acting on particle $i$ is given by:
\[
  \mathbf{f}_i = \mathbf{f}^{pressure}_i + \mathbf{f}^{viscosity}_i
\]

\subsubsection{Computing Velocity and Position}

The acceleration of a particle is computed using Newton's second law:
\[
  \mathbf{a}_i = \frac{\mathbf{f}_i}{m_i}
\]

The velocity and position of the particle are updated using the following equations:
\[
  \mathbf{v}_i = \mathbf{v}_i + \mathbf{a}_i \Delta t
\]
\[
  \mathbf{x}_i = \mathbf{x}_i + \mathbf{v}_i \Delta t
\]
Where $\mathbf{v}_i$ is the velocity of particle $i$, $\mathbf{x}_i$ is the position of particle $i$, and $\Delta t$ is the time step of the simulation.

\subsubsection{Optimization}

The SPH simulation can be computationally expensive, especially for large numbers of particles.
To optimize the simulation, we can use the Spartial hashing technique to efficiently find neighboring particles.

Spartial hashing is a technique that divides the simulation space into a grid of cells, where each cell contains list of particles that are located in that cell.
When computing the neighboring particles for a give particle, we only need to check the particles in the same cell and the neighboring cells.
This reduces the number of particle interactions that need to be computed, resulting in a significant performance improvement.

\section{Rendering}

We render the simulation using OpenGL, which allows for real-time rendering of the fluid dynamics.
In each frame, we assume that the time step is constant, and we update the simulation using the SPH equations.

\medskip

Our implementation uses double buffering to avoid flickering and tearing in the rendering.
In our implementation, the particles are rendered as spheres which created by the shader program.
The shader program is responsible for computing the color and lighting of the particles based on their properties.

\nocite{*}
\printbibliography

\pagebreak
\section*{Appendix}

Our source code is available at \href{https://github.com/MasterIceZ/SPH-FluidSimulation}{Github} and the video demo is available at \href{}{Youtube}.

\end{document}